\documentclass[11pt]{article}
%formattage
\usepackage[left=3cm, right=3cm, top=2.5cm, bottom=2.5cm]{geometry}
\usepackage{xcolor}
\usepackage[T1]{fontenc}
\usepackage{relsize} % mathlarger

\usepackage{fancyhdr}

%fr
\usepackage[utf8]{inputenc}
\usepackage[french]{babel}
\frenchspacing

%math
\usepackage{amsmath, amsfonts}

\usepackage{dsfont} % indicator function : mathds 

% autres
\usepackage{graphicx}
\usepackage{pdflscape} % begin{landscape} \end{landscape}
\usepackage{parskip} % enleve les alinéas et mettre des espaces entre les paragraphes


%%% commandes %%%
\newcommand*{\red}[1]{\textcolor{red}{#1}}
\newcommand*{\blue}[1]{\textcolor{blue}{#1}}
\newcommand*{\purple}[1]{\textcolor{purple}{#1}}
\newcommand*{\green}[1]{\textcolor{green}{#1}}

\newcommand*{\var}[1]{\mathbb{V}\left[{#1}\right]} % renew au lieu de new car le package physics contient déjà \var
\newcommand*{\cov}{\mathbb{C}\text{ov}}
\newcommand*{\corr}{\mathbb{C}\text{orr}} 
\newcommand*{\normal}{\mathcal{N}}
\newcommand*{\uniform}{\mathcal{U}}


\newcommand*{\ehat}[1]{\widehat{\mathbb{E}}\left[{#1}\right]}
\newcommand*{\e}[1]{\mathbb{E}\left[{#1}\right]}
\newcommand*{\p}[1]{\mathbb{P}\left[{#1}\right]}

\newcommand{\indicator}[1]{\mathds{1}_{\{#1\}}}

%% Options de mise en forme du mode français de babel. Consulter la
%% documentation du paquetage babel pour les options disponibles.
%% Désactiver (effacer ou mettre en commentaire) si l'option
%% 'nobabel' est spécifiée au chargement de la classe.
\frenchbsetup{%
	StandardItemizeEnv=true,       % format standard des listes
	StandardItemLabels=true,       % format standard des puces lors d'énumérations
	ThinSpaceInFrenchNumbers=true, % espace fine dans les nombres
	og=«, fg=»,                    % caractères « et » sont les guillemets
	SmallCapsFigTabCaptions=false, % retire les noms des tableaux/figures en petites capitales (je n'aime vraiment pas ça!!!)
}
\addto\captionsfrench{\def\frenchtablename{Tableau}}


\begin{document}

\pagestyle{fancy}
\fancyhead[R]{Automne 2022}
\fancyhead[L]{Aide-mémoire STT-2000}
\fancyfoot[C]{\thepage}


\section{Fondements probabilistes}
\textbf{Définition de la fonction $\Gamma(p)$.} $\forall p > 0, \Gamma(p) = \int_0^\infty x^{p-1}e^x \, dx$.

\textbf{Propriétés de la fonction $\Gamma(p)$.}
\begin{itemize}
	\item $\forall p > 0, \Gamma(p) = (p-1)\Gamma(p-1)$
	\item $\forall p \in \mathbb{N}^*, \Gamma(p) = (p-1)!$
	\item $\Gamma(1)=1$
	\item $\Gamma(1/2)=\sqrt{\pi}$
\end{itemize}

\textbf{Définition de la fonction bêta.}  $\forall x,y<0, \beta(x,y)=\frac{\Gamma(x)\Gamma(y)}{\Gamma(x+y)}$.

\textbf{Fonction génératrice des moments (f.g.m.)} $M_X(t)=\e{e^{tX}} \, \forall t \in \mathbb{R}$ t.q. l'espérance existe.

\textbf{Propriété de la f.g.m.} $\e{X^k} = \left.\frac{d^k}{dt^k}M_X(t)\right|_{t=0}$

\input{2-lois-échantillonnales}

\cleardoublepage
\begin{landscape}
	\thispagestyle{empty}
	
	\begin{table}[]
		\centering
		\resizebox{1.5\textwidth}{!}{%
		\renewcommand{\arraystretch}{2.5}
		\begin{tabular}{c|c|c|c|c}
			$X$ & $f_X(x)$ & $\e{X}$ & $\var{X}$ & $M_X(t)$  \\ \hline \hline
			$\mathcal{N}(\mu,\sigma^2);\mu \in \mathbb{R},\sigma^2>0$ & $\frac{1}{\sqrt{2\pi\sigma^2}}\exp\left(-\frac{(x-\mu)^2}{2\sigma^2}
			\right)\indicator{x\in\mathbb{R}}$ & $\mu$ & $\sigma^2$ & $\exp\left( \mu t + \frac{1}{2}\sigma^2t^2 \right)$ \\ \hline
			$\mathcal{N}_p(\mathbf{m},\sigma^2\boldsymbol{\Sigma});\mathbf{m} \in \mathbb{R}^p,\sigma^2>0$ & $\frac{1}{(2\pi\sigma^2)^{p/2}|\boldsymbol{\Sigma}^{1/2}|}\sigma^{-p}\exp\left(-\frac{(\boldsymbol{x}-\mathbf{m})^\top\boldsymbol{\Sigma}^{-1}(\boldsymbol{x}-\mathbf{m})}{2\sigma^2}\right) \indicator{x\in\mathbb{R}^p}$ & $\mathbf{m}$ & $\sigma^2\boldsymbol{\Sigma}$ & $\exp\left( \mathbf{m}^\top \boldsymbol{t} + \frac{1}{2}\sigma^2\boldsymbol{t}^\top\boldsymbol{\Sigma}\boldsymbol{t} \right)$ \\ \hline
			$t_r; r \in \mathbb{N}^*$ & $\frac{\Gamma((r+1)/2)}{\Gamma(r/2)}\frac{1}{\sqrt{r\pi}}\frac{1}{(1+x^2/r)^{(r+1)/2}} \indicator{x\in\mathbb{R}}$ & 0 si $r>1$ & $\frac{r}{r-2}$ si $r>2$ & -- \\ \hline
			Exponentielle$(\lambda);\lambda>0$ & $ \lambda e^{-\lambda x} \indicator{x>0}$ & $\frac{1}{\lambda}$ & $\frac{1}{\lambda^2}$ & $\frac{\lambda}{\lambda-t}; t < \lambda$ \\ \hline
			Gamma$(\alpha,\lambda); \alpha>0, \lambda>0$ & $\frac{\lambda^\alpha}{\Gamma(\alpha)}x^{\alpha-1}e^{-\lambda x} \indicator{x>0}$ & $\frac{\alpha}{\lambda}$ & $\frac{\alpha}{\lambda^2}$ & $\left( \frac{\lambda}{\lambda-t} \right)^\alpha ;t < \lambda$ \\ \hline
			$\chi^2_k; k \in \mathbb{N}^*$ & $\frac{1}{\Gamma(k/2)2^{k/2}}x^{(k/2)-1}e^{-x/2} \indicator{x>0}$ & $k$ & $2k$ & $(1-2t)^{-k/2}; t<1/2$ \\ \hline 
			$ F_{n,m}; n \in \mathbb{N}^* \in m \in \mathbb{N}^*$ & $\frac{\Gamma((n+m)/2)(n/m)^{n/2}}{\Gamma(n/2)\Gamma(m/2)}\frac{x^{n/2-1}}{(1+nx/m)^{(n+m)/2}} \indicator{x>0}$ & $\frac{m}{m-2}$ si $m>2$ & $\frac{2m^2(n+m-2)}{n(m-2)^2(m-4)}$ si $m>4$ & -- \\ \hline 
			Beta$(\alpha,\beta); \alpha>0,\beta>0$ & $ \frac{\Gamma(\alpha+\beta)}{\Gamma(\alpha)\Gamma(\beta)}x^{\alpha-1}(1-x)^{\beta-1} \indicator{0<x<1}$ & $\frac{\alpha}{\alpha+\beta}$ &  $\frac{\alpha\beta}{(\alpha+\beta)^2(\alpha+\beta+1)}$ & -- \\ \hline
			Uniforme$(a,b); (a,b) \in \mathbb{R}^2$ & $\frac{1}{b-a} \indicator{a<x<b}$ & $\frac{a+b}{2}$ & $\frac{(b-a)^2}{12}$ & $\frac{e^{tb}-e^{ta}}{t(b-a)}$ \\ \hline \hline
		\end{tabular}%
		}%
		\caption{Distribution de certaines lois continues}
		\label{tab:lois-continues}
	\end{table}
	
	\cleardoublepage
	\thispagestyle{empty}
	\begin{table}[]
		\centering
		\resizebox{1.3\textwidth}{!}{%
			\renewcommand{\arraystretch}{2}
			\begin{tabular}{c|c|c|c|c}
				$X$ & $p_X(x)$ & $\e{X}$ & $\var{X}$ & $M_X(t)$  \\ \hline \hline
				Bernoulli$(p)$ & $p^x(1-p)^{1-x}\indicator{x\in\{0,1\}}$ & $p$ & $p(1-p)$ & $pe^t + 1-p$ \\ \hline
				Binomiale$(n,p)$ & ${n \choose x} (1-p)^{n-x} p^x \indicator{x \in \mathbb{N}}$ & $np$ & $np(1-p)$ & $(pe^t + 1-p)^n$ \\ \hline 
				Géométrique$(p)$ (échec) & $p(1-p)^{x} \indicator{x \in \mathbb{N}}$ & $\frac{1}{p}-1$ & $\frac{1-p}{p^2}$ & $\frac{p}{1-(1-p)e^t}$ \\ \hline
				Géométrique$(p)$ (succès) & $p(1-p)^{x-1} \indicator{x \in \mathbb{N^*}}$ & $\frac{1}{p}$ & $\frac{1-p}{p^2}$ & $\frac{pe^t}{1-(1-p)e^t}$ \\ \hline
				Uniforme$(m)$ & $\frac{1}{m} \indicator{x \in \{1,\dots,m\}}$ & $\frac{m+1}{2}$ & $\frac{m^2-1}{12}$ & $\frac{e^t(e^{mt}-1)}{e^t-1}$ \\ \hline
				Poisson$(\lambda)$ & $e^{-\lambda} \lambda^x/x! \indicator{x \in \mathbb{N}}$ & $\lambda$ & $\lambda$ & $\exp(\lambda(e^t-1))$ \\ \hline\hline
			\end{tabular}%
		}
		\caption{Distribution de certaines lois discrètes}
		\label{tab:lois-discretes}
	\end{table}
\end{landscape}

\end{document}